\documentclass[a4paper]{article}

\usepackage[swedish]{babel}

\usepackage{amsthm}
\usepackage{amssymb}
\newtheorem{theorem}{Sats}[section]
\newtheorem{definition}[theorem]{Definition}
\newtheorem{lemma}[theorem]{Lemma}
\newtheorem{proposition}[theorem]{Proposition}
\newtheorem{corollary}[theorem]{Korollarium}
\newtheorem{example}[theorem]{Exempel}

\author{John Paul Edward Möller}
\title{Lösning av 3-SAT med neurala nätverk}

\begin{document}
\maketitle

  \section{Introduktion} \label{sec:Introduktion}
  här står det saker

 \section{Komplexitet} \label{sec:Komplexitet}
 
 \subsection{P vs NP}%
 \label{sub:pnp}
 
 'P vs NP' eller 'P = NP' är ett Millenium problem, d.v.s ett av sju
 problem som Clay Institutet erbjuder en miljon dollar för en lösning.
 Det behandlar om de två komplexitetsklasserna 
 
 \subsection{Propositionell Logik}%
 \label{sub:prop_log}

 I gymnasial matematik behandlar man främst variabler som ingår i oändliga
 mängder såsom de naturliga, reella och komplexa talen. Men det kan vara
 väldigt användbart att använda sig av matematik som behandlar variabler som
 ingår i enkla mängder. Dock så vill man forfarande att det ska vara användbart.
 Den tomma mängden och en mängd med ett element är enkla med de är inte särskillt
 användbara; en variabel kan ej ingå i en tom mängd och en variabel i en mängd
 med ett element är ju redan löst. Så vi kommer betrakta matematik som
 behandlar variabler som ingår i en mängd med två element. Sådan matematik
 kallas Propositionell logik.
 
 \begin{definition}
 \label{def:bool_var}
 Ett element i mängden $\mathbb{B} = \left\{ 0,1 \right\} $ kallas för ett
 booleskt värde.
 \end{definition}

 Likt de tidigare mängderna, kan $0$ och  $1$ symbolisera kvantiteter, men
 styrkan med booleska variabler är att $0$ och $1$ kan symbolisera saker som
 är i motsats till varandra. Detta kan vara falskt och sant, av och på, ner och
 upp o.s.v..  När propositionell logik diskuteras i fortsättningen av denna text
 kommer orden 'falskt' och 'sant' användas syonymt med siffrorna $0$ och $1$ respektivt.
 
 Med mängderna som de naturliga, reella och komplexa talen fanns det en tydlig
 betydelse av operationerna plus,minus,gånger och division. Detta är dock inte
 lika tydligt med booleska värden. Istället defineras följande två
 operationer.

 \begin{definition}
 \label{def:and}
 Om $A,B \in \mathbb{B}$ så är $A * B  $ lika med $1$ om och endast om $A$ och $B$ 
 är båda lika med $1$, annars är det lika med $0$. $A * B$ utalas 'A och B'.
 \end{definition}

 \begin{definition}
 \label{def:or}
 Om $A,B \in \mathbb{B}$ så är $A + B$ lika med $0$ om och endast om både $A$
 och $B$ är lika med $0$, annars är det lika med $1$. $A+B$ utalas 'A eller B'.
 \end{definition}

 \begin{example}
 \label{eg:sky}
 Låt påståendet 'Himlen är ibland blå' betecknas med variabeln $A$, låt påståendet
 'Himlen är ibland svart' betecknas med variabeln $B$ och låt påståendet 'Himlen
 är ibland grön' betecknas med variabeln $C$. Alltså:
 \[
   A=1,B=1,C=0
 .\] 

 'Himlen är ibland blå och himlen är ibland svart' är då sant och kan uttryckas
 som:
 \[
 A*B = 1
 .\] 

 'Himlen är ibland blå och himlen är ibland grön' är då falskt och kan
 uttryckas som:
 \[
 A*C = 0
 .\] 

 'Himlen är ibland blå eller himlen är ibland grön' är då sant och kan
 uttryckas som:
 \[
 A+C = 1
 .\] 
 \end{example}


 Som man ser i exempel \ref{eg:sky} så är definitionen 'och' ganska synonymt
 till hur man använder ordet i vardagligt språk. Sista påståendet som använde
 'eller' översätts inte lika tydligt i vissa vardagliga uttryck, då det kan uppfattas
 som en fråga. Men i matematiska sammanhang är det alltid ett påstående.

 \begin{definition}
 \label{def:dis}
 En formel som är bara kopplade med 'eller' d.v.s på formen $a_1 \land  a_2
 \wedge  a_3\ldots$ upp till $n$, kallas för en disjunktion.
 \end{definition}
 

 

 \subsection{3-SAT}%
 \label{sub:3sat}



 
 

 
 
 
 
 


\end{document}
